\chapter{Introduction}

Almost everyone uses the internet everyday, whether connecting to the internet from their mobile devices or browsing the internet on their personal computers. It is safe to say that most of us will struggle with everyday activities without devices that are capable to connect to the internet. For example, getting a taxi ride, ordering food delivery or even navigating to places we want to go. Apple have made an interesting short movie "predicting" what might happen without Apps on mobile devices during their annual Worldwide Developers Conference (WWDC) in 2017 [15].
 
Internet have an even greater presents and role within the science and engineering community, sites like Google Scholar and The New England Journal of Medicine have benefited scientists, doctors and engineers greatly [15] [16] [17]. Prior to the wide adaption of the internet, searchers, scholars and students would spend countless hours in physical libraries, browsing through hundreds of thousands of academic journals and books in order to answer one question or to solidify a proposal they had in mind. Today with the help of the internet, it will hardly take a few minutes. This leads to the jump in research productivity which helps researchers and engineers to produce their research and production output to a new level, thus allowing them to essentially do more things with the same amount of time.
 
For the last 20 years, PHP and JavaScript are the technologies that dominates the majority of web browsers and web development. Although updates and new versions are issued regularly, they are still becoming more aged everyday, and software engineers have been finding ways to improve the performance, user experience and developer experience for this ageing technology. For example, Microsoft developed the Typescript language to combat the "strange" nature of the JavaScript language [18], and to help resolving the lack of type checking within JavaScript [19]. Facebook developed the "React" library framework to improve developer experience by introducing the "component" concept. Where each class or function within a React project can be seen as a UI (user interface) component. This can be a button, a table or a component with more complexity, such as a navigation header component [20]. As expected and intended, some of the popular web development frameworks and libraries such as React JS and Angular (Google) have gained significant attention and much of the industry-wide adaption since their release and they have indeed significantly improved browsing experience for all of us.
 
However, little did we know, a major change was just around the corner.
 
\bigskip
\bigskip

\textbf{{\Large Chapter 1.1 Meet WebAssembly}}

\bigskip

WebAssembly was first released and adopted in 2017 by a number of browsers [21]. Since then, there has been a growing community built around the technology as more and more projects started using WebAssembly. As mentioned above, JavaScript have been the de facto language for web browsers since the beginning of the web. However, this was never the original intention during the initial development of JavaScript.

There is a wide spread misconception that Al Gore, the Former Vice President of the United States, invented the internet. However, this is not true as Al Gore has never claimed or said so. Instead, what he actually said was "I took the \textbf{initiative} in creating the internet" during his interview with CNN in 1999 [22]. However, AL Gore did introduced the High Performance Computing Act of 1991 [23], which help funded the creation of the first mainstream web browser Mosaic [24], as well as the creation of the high-speed fiber optic computer network [25].

