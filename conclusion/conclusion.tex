\chapter{Conclusion} \label{chap:conclusion}

In this thesis, we studied and explored a relatively new research topic within the field of Computer Science, WebAssembly. Furthermore, we looked into a very niche area within WebAssembly: \textbf{the use case of the technology running as microservices for server-side computing.} We collected and presented some of the current work other researchers and developers have undertaken for both WebAssembly running in the frontend client-side as well as server-side backend implementation. We then designed and conducted various experiments to prove or disprove our initial theory and hypothesis. More specifically, we undertook our investigation in two stages. The first stage compares the WebAssembly framework solution with a non-WebAssembly traditional framework implementation undertaking benchmark tests across our sorting algorithms collection, which includes \textbf{18} tests. The second stage of the experiment involves \textbf{4} more shortest path/minimum spanning tree benchmarking algorithms. Finally, we collected and analysed the data collected from the experiments and delivered our own verdict on the current state of server-side WebAssembly solution development, especially in the commercial sector.

We also discovered the impressive growth rate for WebAssembly running in web browsers and backend servers. Especially with the letter in which the technology was only made available as recently as \textbf{2019}. Therefore, the technology has significant potential in terms of runtime speed as well as runtime efficiency. This could lead to more industry adoption of the technology in the future and open up new possibilities, research areas and markets. For example, the ability for WebAssembly to run on edge/IoT devices with a much lower hardware requirement can lower the cost of processing computers in \textbf{automobiles}. And its lightweight, low-latency features can contribute to and solve some of the most critical problems within the self-driving vehicle industry.

Overall, the results from the experiment are somewhat surprising given the previous track record for the technology from the literature review. Nonetheless, we have collected valuable data from the investigation, tested the limitations of the current WebAssembly technology running on the edge, and we can justify the results upon a few external factors. Such as the current limitations of Paas services as well as the runtime performance for the underlying programming language. On top of that, we have also proposed possible further research directions and experiments. The experiments conducted in this thesis varies significantly from prior research and studies. While other studies focus primarily on "paper performance", we created a full-stack solution in which we designed and developed a complete benchmarking application with proper production deployed on \textbf{Amazon Web Service}. We found no prior studies comparing the two framework solutions in a commercial sense, and we are confident that our result will be used in the decision-making for future WebAssembly framework products' development.