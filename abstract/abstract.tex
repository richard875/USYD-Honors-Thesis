\chapter*{Abstract}

From the creation of the first computer back in 1944 \cite{abs1}, to the wide adoption of the internet that led to the "dot com bubble" in the late 90s \cite{abs2}, all the way to today, where everyone and everything are connected. It is hard to believe we used to live in a time without any of these. It is even harder to imagine what our life would be like without all the technologies and devices around us. Humanity has indeed come a long way since these days.

Many of us have been around since the early days of the internet, growing up during the age of AOL dial-up services and browsing the internet at a maximum speed of 56 kilobytes per second. Therefore, it is magnificent to see and experience the magic of modern web technology, where websites are no longer just pre-written pages downloaded from the server but complete interactive web applications that can book hotels \cite{abs3}, shop online \cite{abs4} and even create complex design works \cite{abs5}.

Just a few years ago, in 2017, a new web browser technology standard called \textbf{WebAssembly} \textbf{(WASM)}, was developed and introduced \cite{abs6}. WebAssembly has the unique ability to run and execute desktop application code such as c++ and Java in web browsers. Enabling web pages to act more like desktop applications with greater access to the system APIs that were previously only available to native desktop applications while enhancing the current safety and security standards adopted by all modern web browsers. WebAssembly achieves this by compiling those traditional desktop application languages down to WebAssembly code; the WASM code will then be interpreted directly by the supported browser without needing to be further converted into JavaScript. Ever since the introduction of WebAssembly, a number of exceptional projects have been developed by companies and individuals \cite{abs7}. For example, \textbf{Figma} \cite{abs5} built their high-performance document editing engine in c++ and by compiling the code to WebAssembly, they were able to achieve \textbf{3} times faster document load time compared to the previous rendering technology \cite{abs8}. \textbf{Adobe} also announced in late 2021 that it has started to roll out a web version of its Photoshop and Illustrator product \cite{abs9}\cite{abs10}. WebAssembly is what made it all possible \cite{abs11}.

In this thesis, we will introduce a more recent use case of running WebAssembly outside web browsers - WebAssembly running on edge devices. We will also justify the reason for individuals to choose this new technology over the current way of implementation. Although WebAssembly was first developed and designed to be used in web browsers, the release of WebAssembly System Interface (WASI) in March 2018 made it possible to run WASM code directly from operating systems through a runtime \cite{abs12}.

Since then, researchers and engineers have been working to develop software of all aspects with WASM runtimes. From internet of things (IoT) frameworks to operating systems. As the technology gets more mature and widely adapted, its overhead will reduce over time, thus further improving the performance of the WebAssembly language \cite{abs13}.

As a part of our research, we will be looking into the current state and development of WebAssembly frameworks running directly on edge devices, as well as its performance gap with the traditional implementations, which usually consist of applications running on VM (virtual machine) containers.

We will first gain an understanding of WebAssembly and the current state of adoption for the technology. We will then design and undertake experiments to evaluate and analyse our experiments' performance and other aspects. After that, we will launch a discussion on the topic based on the results produced by those experiments. Finally, we will provide a verdict on our research topic as well as present questions and projects that have the potential for further research.

\bigskip

In this thesis, we will use the terms \textbf{WASM} and \textbf{WebAssembly} interchangeably.