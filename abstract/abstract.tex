\chapter*{Abstract}

Since the creation of the first computer back in 1943 [1], to the widely adoption of the internet that lead to the "dot com bubble" in the late 90s [2], all the way to today where everyone and everything are connected. It is hard to believe we used to live in a time without any of these. It is even harder to imagine what our life will be like without all the technologies and devices around us. The humanity have truly came a long way since these days.

Many of us have been around since the early days of the internet, growing up during the age of AOL dial up services and browsing the internet at the maximum speed of 56 kilobit per second. It is magnificent to see and experience the magic of the modern web technology where websites are no long just pre-written pages downloaded from the server, but full web applications that are able to book hotels [3], shop online [4] and even create complex design works [5].

Just a few years ago in 2017, a new web browser technology standard call \textbf{WebAssembly} \textbf{(WASM)}, was developed and introduced. WebAssembly have the unique ability to run and execute desktop application code such as c++ and Java in web browsers. Enabling web pages to act more like desktop applications with greater access to the system APIs that was previously only available to native desktop apps while enhancing the current safety and security standards adopted by all modern web browsers. WebAssembly achieves this by compiling those traditional desktop application languages down to WebAssembly code, the WASM code will then be interpreted directly by the supported browser without needing to be further converted into JavaScript. Ever since the introduction of WebAssembly, a number of exceptional projects have been developed by companies and individuals [7]. For example, \textbf{Figma} [5] built their high-performance document editing engine in c++ and by compiling the code to WebAssembly, they were able to achieve 3 times faster document load time compare to the previous rendering technology [9]. \textbf{Adobe} also announced in late 2021 that it has started to roll out a web version for its Photoshop and Illustrator product [9][10]. WebAssembly is what made it all possible [11].

In this thesis, we will introduce a more recent use case of running WebAssembly outside of web browsers - WebAssembly running on edge devices. Although WebAssembly was first developed and designed to be used on web browsers, the release of WebAssembly System Interface (WASI) in March 2018 made it possible to run WASM code directly from operating systems through a runtime [12].

Since then, researchers and engineers have been working to develop software of all aspects with WASM runtimes. From Internet of things (IoT) frameworks to operating systems. As the technology gets more matured and widely adapted, the size of its overhead will reduce overtime, thus further improving the performance of the WebAssembly language [13].

As a part of our research, we will be looking into the performance gap and performance inconsistency between the current way of running applications on the edge devices - c/c++ framework running on VM (virtual machine) containers, and a more modern solution with WebAssembly running directly on edge devices as individual functions.

We will first gain an understanding in WebAssembly and the current state of adoption for the technology. We will then design and undertake experiments to evaluate and analyse the performance and other aspects within our experiments. After that, we will undertake discussion on the topic based on the results produced by the experiments. Finally, we will provide a conclusion to our research topic as well as advising questions and projects that has the potential to conduct further research in.